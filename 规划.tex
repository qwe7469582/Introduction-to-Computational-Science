\documentclass{article}
\usepackage[UTF8]{ctex}
\usepackage{geometry}
\usepackage{multirow}
\usepackage{natbib}
\geometry{left=3.18cm,right=3.18cm,top=2.54cm,bottom=2.54cm}
\usepackage{graphicx}
\pagestyle{plain}	
\usepackage{setspace}
\usepackage{enumerate}
\usepackage{caption2}
\usepackage{datetime} %日期
\renewcommand{\today}{\number\year 年 \number\month 月 \number\day 日}
\renewcommand{\captionlabelfont}{\small}
\renewcommand{\captionfont}{\small}
\begin{document}

\begin{figure}
    \centering
    \includegraphics[width=8cm]{upc.png}

    \label{figupc}
\end{figure}

	\begin{center}
		\quad \\
		\quad \\
		\heiti \fontsize{45}{17} \quad \quad \quad 
		\vskip 1.5cm
		\heiti \zihao{2} 《计算科学导论》个人职业规划
	\end{center}
	\vskip 2.0cm
		
	\begin{quotation}
% 	\begin{center}
		\doublespacing
		
        \zihao{4}\par\setlength\parindent{7em}
		\quad 

		学生姓名:\underline{\qquad  朱宜昌 \qquad \quad}

		学\hspace{0.61cm} 号:\underline{\qquad 1709030129\qquad}
		
		专业班级:\underline{\qquad 物理1701 \qquad  }
		
        学\hspace{0.61cm} 院:\underline{\qquad \quad 理学院 \qquad \quad}
% 	\end{center}
		\vskip 1.5cm
		\centering
		\begin{table}[h]
            \centering 
            \zihao{4}
            \begin{tabular}{|c|c|c|c|c|c|c|c|c|}
            % 这里的rl 与表格对应可以看到,姓名是r,右对齐的;学号是l,左对齐的;若想居中,使用c关键字。
                \hline
                \multicolumn{5}{|c|}{分项评价} &\multicolumn{2}{c|}{整体评价}  & 总    分 & 评 阅 教 师\\
                \hline
                自我 & 环境 & 职业 & 实施 & 评估与 & 完整性 & 可行性 &\multirow{2}*{} &\multirow{2}*{}\\
                分析& 分析& 定位 & 方案 & 调整 & 20\% & 20\% & ~&~ \\\            
                10\% & 10\% & 15\% & 15\% & 10\% & &  &~ &~\\
                \cline{1-7} 
                & & & & & & & ~&~ \\
                & & & & & & & ~&~ \\
                \hline      
            \end{tabular}
        \end{table}
		\vskip 2cm
		\today
	\end{quotation}

\thispagestyle{empty}
\newpage
\setcounter{page}{1}
% 在这之前是封面,在这之后是正文
\section{自我分析}
	自我分析即对自己进行全方位、多角度的分析,目的是认识自己、了解自己。只有认识了自己,才能对自己的职业做出正确的选择,才能选定适合自己发展的职业生涯路线,才能对自己的职业生涯目标做出最佳抉择。\par
	自我分析包括:\par
\subsection{自然条件}
男生,目前21岁,身体健康状况良好,有一定的学习精力和能力。目前在青岛上大学,家住河南,家庭条件一般。\par
\subsection{性格分析}
经过自我MBTI人格测试,结合自己对自己的认识,得出自己是ENFP型人格。对于ENFP型人格有如下特点:

1.充满热忱、活力充沛、聪明的、富想象力的,视生命充满机会但期能得自他人肯定与支持。

2.几乎能达成所有有兴趣的事。

3.对难题很快就有对策并能对有困难的人施予援手。

4.依赖能改善的能力而无须预作规划准备。

5.为达目的常能找出强制自己为之的理由。

6.即兴执行者。

对于自己感兴趣的事情会有很高的执行力,但是同时也希望得到他认得支持和认可,善于伸出援手,在某些方面会有偏执。另外个人感觉自己在对待选择上有一定的纠结,有些事心里想但是常常不敢迈出第一步,目前仍在进一步的了解自己。\par
\subsection{教育与学习经历}
中国石油大学(华东)本科在读。\par
\subsection{工作与社会阅历}
无。\par
\subsection{知识、技能与经验}
熟悉C语言、Java、Python爬虫、掌握部分机器学习算法和数据处理方法。曾参加全国数学建模竞赛、山东省光电设计竞赛等,获得两项省级奖项。曾参与项目:图书馆管理系统、坦克大战游戏开发、华容道图形界面游戏开发等。熟练使用Office软件,熟悉Origin、Matlab等数据处理软件.一篇科技核心论文已投稿并录用。\par
\subsection{兴趣爱好与特长}
编程、篮球、唱歌、街舞等。\par
\section{环境分析}
环境分析主要是评估周边各种环境因素对自己职业生涯发展的影响。每一个人都处在一定的环境之中,职业发展必然要受到所处环境的影响,只有充分了解和把握所处环境的现状、特点、发展变化趋势,才能做到在复杂的环境中避害趋利,使你的职业生涯规划具有实际意义。\par
环境分析包括:\par
\subsection{社会环境分析}
国家目前大力支持高新技术的发展,高技术人才得到越来越大的重视,对当代科技工作者来说是一个很好的时代。另外随着当前人工智能热潮的出现,计算机相关技术占有越来越大的市场,目前仍具有很好的发展前景。

计算机现在的就业情况不比以前乐观,但可以说仍然是最热门的专业之一。本科毕业一般就写代码,算是IT的底层,工作辛苦不说,工资其实也不算多。以后做项目也会很忙,但工资会涨幅比较快(根据能力)。

当前社会,学计算机的人很多,但是人才不多,特别是高级人才不多,所以说计算机行业竞争激烈是针对普通初级人才而言的。近年来,本科段的计算机科学与技术专业的就业率一直就不高,而且有下降趋势,这主要是由于高校扩招和高校普遍设置该专业造成的人才过剩。即便是最优秀的学府出来的人也良莠不齐。\par
\subsection{家庭环境分析}
目前大学本科在读,未婚,家住农村,家庭经济情况一般。姐姐目前研究生在读,家人希望工作能离家近一点,能够经常回家。个人意愿是研究生毕业之后能在家乡附近工作,研究生去附近的城市深造。\par
\subsection{职业环境分析}
计算机科学技术包括很多发展方向:一是软件编程,这个就要和程序打交道,虽然枯燥但很有前途。二是网络工程,主要是网络构建基本的网络知识。三是硬件,就是计算机的具体构造,各个部件的联系工作原理,这个专业学的东西比较广,以上三个方向都会涉及。选择一个主攻方向对今后的发展很重要,比如侧重物理硬件的偏硬方向及数学逻辑的软件方向,还有和大家关系密切的商务方向,等等。

从IT行业的前景说,海外软件外包业务可谓热火朝天,国外很多发达国家会把他们要开发的软件放到中国做,因为中国的成本相对来说比较低一些,现在这个势头在中国还是刚刚开始,所以,未来十年内,软件开发这个领域的本科生是非常好找工作的,如果你技术精通,英语、日语或者韩语流利,那么成为一个超级金领不成问题,或者如果自己愿意创业开公司,难度和风险相对其他专业都要小很多。

近年来,本科毕业就业人数不高。但是计算机在用人单位心中的印象还是可以的,这个专业的就业率应该说是比较高的,但对于毕业生来说,学校每年在毕业前夕会组织多项毕业生洽谈会,用人单位会根据其所需招些合格的毕业生到他们单位实习,收入当然肯定会能力、技术挂钩。\par


\subsection{地域与人际环境分析}
目前在青岛读大学,附近计算机相关企业较多,作为新一线城市,青岛在就业方面有一定的优势。人际环境方面,学校有很丰富的师资力量,老师们有着很高的专业水平,可以很方便的向老师和同学咨询相关事宜。\par
\par 
\iffalse
图片插入的样例:\par
\begin{figure}[h!]
\centering
\includegraphics[scale=1.7]{universe}
\caption{The Universe}
\label{fig:universe}
\end{figure}
\fi


\section{职业定位}
在准确地对自己和环境做出了分析之后,确定适合自己行业和有实现可能的职业发展目标。职业定位时要注意与自己的自然条件、知识背景、技能特长、性格特点、兴趣爱好是否匹配,考虑与自己所处的环境是否相适应。职业定位决定了职业发展中的行为和结果,是制定职业生涯规划的关键,应当科学合理,具有可行性。\par
职业定位包括:\par

\subsection{行业领域定位与理由}
由技术性到管理型,若研究生毕业则努力去做自然语言处理或者计算机视觉等相关工作。理由:个人偏爱于算法分析与数据处理,另外在技术之外,个人比较喜欢与人打交道,希望之后能从事技术管理类的工作。\par
\subsection{职业岗位起点定位与理由}
本科毕业若工作较难找到人工智能相关的工作,若本科毕业工作则从事Java后端开发相关工作,另外争取找到人工智能相关岗位。若正常读研,则在研究生时积累项目经验,毕业时第一份工作能够在项目中起到较关键的作用。\par
\subsection{职业目标与可行性分析}
\par
成果目标、经济目标、能力目标、职务目标等。\par 
\begin{enumerate}[(1)]
	\item 短期目标(大学4年)
	
成功毕业拿到学位证争取保研,参与到老师的科研项目中去,提前了解专业前沿知识,提高自己的能力。英语六级成绩达到500分,具备英文文献的阅读能力。拿到三个以上省级以上比赛的奖项,拿到数学建模或程序设计类竞赛国家级奖项。如有条件争取发表论文。
	\item 中长期目标(5-10年)。
	
研究生考入985院校计算机学硕,顺利毕业,之后参加工作从事自己满意的岗位,实现经济独立并且能拿到自己的期望薪资,工作地点能够在家乡附近,想回家的时候可以较为方便。在毕业的时候能有较高的专业素养和专业能力,在工作中能够有一定的话语权,生活工作基本稳定,成家立业。
\end{enumerate}
\iffalse
这里是简单列表的样例:(如果需要标号自定义或者自动标记数字序号,请自行搜索语法)

\begin{itemize}
    \item 简单的列表结构 
    \item 如这里所示
    \item 此处仅为样例
    \item 按需修改和使用
\end{itemize}
\fi

\section{实施方案}
在明确了职业定位后,要制定实现职业生涯目标的行动方案,不付诸行动,职业目标只能是一种梦想。实施方案是实现职业目标的保证,尽量考虑周全、具有可操作性。\par
实施方案可以从以下角度考虑:\par
\begin{enumerate}[1、]
	\item 如何利用现有条件和自身优势以实现职业生涯目标:通过学习专业知识,目前已经具备一定的编程能力,在下学期参与到老师的创新项目中去,进一步了解科研工作,为研究生的生活打下基础。多向老师和学长学姐们请教,了解专业前沿知识和职业发展方向等。
	\item 如何克服缺点、弥补不足、增长知识、提高能力以实现职业生涯目标:个人在性格方面对待事物有部分的纠结,并且不容易在短时间内进入到较好的工作状态,之后需要多注意和锻炼。另外,一直以来身体都不是特别的强壮,需要注意锻炼身体以使得自己有足够的精力去学习和工作。在学习和工作方面,通过了解自己所感兴趣的方向,通过各种渠道了解相关知识,如各大论坛、图书馆等。
	\item 如何处理人际关系和发展人脉以实现职业生涯目标:常怀感恩之心,友好对待日常工作和学习中的伙伴,尊敬师长。积极向自己感兴趣的方向的大佬学习,拓宽自己的视野,结识志同道合的朋友。积极参加相关比赛,在比赛中提高自身能力拓宽交际圈,努力走向更好的平台。
	\item 如何处理工作与家庭、生活的关系以实现职业生涯目标:始终谨记身体是革命的本钱,家人是最亲近的人。在工作中积极进取,保质保量完成自己的任务,工作之余多余家人联系,多参与体育锻炼,注意养成良好的生活习惯。
	\item 如何处理释放工作压力、保证身心健康以实现职业生涯目标:培养个人业余爱好,对我个人来说比较喜欢打篮球和街舞,在学习和工作之余,体育锻炼是放松的最好方法。另外找朋友倾诉,和朋友一起出去散心也是不错的选择。
\end{enumerate}
\par 
\iffalse
表格插入样例(三线表):\par
%单元格怎么写?参考第一页打分的表格


\begin{table}[h]
	\centering
	\caption{这是科学系的花名册}
	\begin{tabular}{rl}
		% 这里的rl 与表格对应可以看到,姓名是r,右对齐的;学号是l,左对齐的;若想居中,使用c关键字。
		\hline
		姓名 & 学号 \\
		\hline
		张三 & 190704xxxx+++ \\ 
		李四 & 190704yyyy \\
		王二五 & 190704zzzz\\
		\hline
	\end{tabular}
	\label{table1}
\end{table}
\fi
\section{评估与调整}
由于影响职业生涯规划的因素很多,且大都处于动态变化之中,因此职业生涯规划应定期评估,并根据影响因素的变化和实施结果的情况及时作出调整,这样才能保证其行之有效。\par 
\subsection{评估时间}
每学年一次。\par
\subsection{评估内容}
每年的既定计划是否完成,所需要完成的任务是否保质保量完成。对完成的总结经验,对未完成的分析原因。\par
\subsection{调整原则}
大方向上不动摇,认真分析当前环境,积极大胆做出改变。\par




\end{document}
